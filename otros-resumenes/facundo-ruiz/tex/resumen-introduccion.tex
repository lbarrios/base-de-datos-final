\section*{Introducción}

\subsection*{Definiciones}
\begin{itemize}
    \item \textbf{Base de datos:} Es un conjunto de datos relacionados con un \textbf{significado inherente}. Esto hace que no consideremos a un conjunto de datos aleatorios como una, ya que se diseña y contruye con un propósito específico. \\
    Surgieron en la década del '70 debido a las crecientes necesidades de almacenar ya acceder a mayores cantidades de datos de forma eficiente. Suelen estar manejadas por un \textbf{DBMS} (Database Management System).
    \item \textbf{Dato:} Es un hecho conocido que puede ser registrado y tiene un significado implícito. Suele ser más volatil que la estructura en que está definido.
    \item \textbf{DML (Data Manipulation Language):} Es el usado en los DBMS para modificar las instancias: obtener, agregar, cambiar, etc. Algunas de sus funciones en SQL son SELECT, INSERT, DELETE de SQL.
    \item \textbf{DDL (Data Definition Language):} Se usa en los DBMS para alterar el esquema de la base: crear, agregar atributo, renombrar. Algunas de sus funciones en SQL son CREATE, ALTER, DROP, RENAME.
\end{itemize}

\subsection*{DBMS}
Es la herramienta que utiliza cada aplicación para manejar grandes cantidades de datos de manera eficiente. Suele ser configurada por los \textbf{administradores de bases de datos}.
Entre sus funciones tenemos:
\begin{enumerate}
    \item Permitirle a los usuarios crear nuevas bases de datos y especificar sus esquemas.
    \item Perimitirle a los usuarios realizar consultas a los datos y modificarlos.
    \item Almacenar grandes cantidades de datos por un período largo de tiempo para perimitir consultas y modificaciones eficientes.
    \item Garantizar la durabilidad de los datos al tener un sistema de recuperación en caso de fallas, errores o mal uso intencional. Este también se encarga de las copias de seguridad.
    \item Controlar el acceso concurrente a los datos por parte de los usuarios, de manera de evitar interacciones inesperadas y garantizar que las acciones realizadas sean seguras y completas.
\end{enumerate}
Otros componentes de la DBMS incluyen:
\begin{itemize}
    \item \textbf{Recovery Manager:} Encargado de restaurar la base de datos a un estado consistente en caso de haber una falla. Para hacer eso hace uso del \textbf{log}, un archivo que lleva un registro de las acciones efectuadas a la base de datos.
    \item \textbf{Optimizador de consultas:} encargado de armar un plan de ejecución eficiente en base a una consulta. Para ello hace uso del \textbf{system catalog}.
\end{itemize}

\subsection*{System catalog}
Es el lugar de un \textbf{RDBMS} (Relational DBMS) en que se guardan los metadatos del esquema. Estos comprenden:
\begin{itemize}
    \item Información sobre tablas y columnas. Esto comprende datos estadísticos como el tamaño de los archivos y el factor de bloqueo, la cantidad de tuplas de la relación, la cantidad de bloques de la relación y el rango de valores de una columna.
    \item Vistas, interpretadas como consultas que pueden guardarse para referenciarse nuevamente.
    \item Índices, utilizados para optimizar las consultas.
    \item Usuarios y grupos de usuarios, para controlar los accesos.
    \item Triggers, para actualizar automáticamente ciertos datos en respuesta a eventos o acciones.
    \item Funciones de agregación definidas por el usuario.
\end{itemize}
Entre sus principales usos tenemos:
\begin{enumerate}[label=\roman*]
    \item Obtener el esquema de una tabla al verificar una consulta.
    \item Obtener la selectividad esperada de un atributo al optimizar una consulta.
    \item Obtener los permisos de un usuario al verificar un acceso.
\end{enumerate}

\subsection*{Arquitectura e independencia}
En una base de datos tenemos 3 \textbf{niveles}:
\begin{itemize}
    \item \textbf{Interno:} Es el que describe el almacenamiento físico de las estructuras de la base de datos.
    \item \textbf{Conceptual (o lógico):} Es el que contiene el esquema conceptual que describe la estructura de la base de datos sin enfocarse en lo físico sino en entidades, tipos de datos, operaciones de usuarios y restricciones.
    \item \textbf{Externo (o de usuario):} Contiene los esquemas o vistas de usuarios en los que se describe la parte de la base de datos que le interese a un grupo en particular.
\end{itemize}
Entre ellos podemos tener las siguientes \textbf{independencias}:
\begin{itemize}
    \item \textbf{Lógica:} Capacidad de poder cambiar el esquema conceptual sin cambiar los externos, ya sea para expandir o reducir la base de datos  (agregar o quitar un nuevo tipo de registro), cambiar las restricciones, etc. No suele ser fácil de lograr.
    \item \textbf{Física:} Capacidad de poder cambiar el esquema interno sin cambiar el concpetual (y por ende el externo). Esto involucra cambiar la organización de los archivos o agregar algún índice para las consultas.
\end{itemize}
